\section{はじめに}

高速フーリエ変換(FFT)のメモ

\section{離散フーリエ変換}

関数 $f(x)$ に対するフーリエ変換・逆フーリエ変換を以下で定義する

\begin{align}
    \tilde{f}(k) &= \int_{-\infty}^{\infty} f(x)e^{-ikx} dx \\
    f(x) &= \frac{1}{2\pi} \int_{-\infty}^{\infty} \tilde{f}(k) e^{ikx} dk
\end{align}

これを離散化して離散フーリエ変換を以下で定義する。

\begin{align}
    \tilde{a}_k &= \sum_{m = 0}^{N - 1} e^{-i\frac{2\pi}{N} km} a_m
\end{align}

一方、逆フーリエ変換は以下で定義する

\begin{align}
    a_m &= \frac{1}{N}\sum_{k = 0}^{N - 1} e^{i\frac{2\pi}{N} km} \tilde{a}_k
\end{align}

ここで$W_N$を以下で定義する

\begin{align}
    W_N \equiv e^{-i\frac{2\pi}{N}}
\end{align}

これにより離散フーリエ変換は以下のように書ける

\begin{align}
\tilde{a}_k &= \sum_{m = 0}^{N - 1} W_N^{km} a_m \\
    a_m &= \frac{1}{N}\sum_{k = 0}^{N - 1} W_N^{-km} \tilde{a}_k
\end{align}
