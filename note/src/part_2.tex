\subsection{具体例}

\subsubsection{$N = 2$ の場合}

離散フーリエ変換は

\begin{align}
    \tilde{a}_k = \sum_{m = 0}^{1} W_2^{mk} a_m
\end{align}

と書けて$W_N$は

\begin{align*}
    W_2 = e^{-i\pi} = -1
\end{align*}

であるので

\begin{align}
    \begin{pmatrix}
        \tilde{a}_0\\
        \tilde{a}_1
    \end{pmatrix}
    =
    \begin{pmatrix}
        1 && 1 \\
        1 && -1
    \end{pmatrix}
    \begin{pmatrix}
        a_0\\
        a_1
    \end{pmatrix}
\end{align}

または

\begin{align}
    \begin{cases}
        \tilde{a}_0 = a_0 + a_1\\
        \tilde{a}_1 = a_0 - a_1
    \end{cases}
\end{align}

\subsubsection{$N = 4$ の場合}

\begin{align}
    \tilde{a}_k = \sum_{m = 0}^{3} W_4^{mk} a_m
\end{align}

と書けて$W_N$は

\begin{align*}
    W_4 = e^{-i\frac{\pi}{2}} = -i
\end{align*}

であるので

\begin{align}
    \begin{pmatrix}
        \tilde{a}_0 \\
        \tilde{a}_1 \\
        \tilde{a}_2 \\
        \tilde{a}_3
    \end{pmatrix}
    =
    \begin{pmatrix}
        1 && 1 && 1 && 1 \\
        1 && -i && -1 && i \\
        1 && -1 && 1 && -1 \\
        1 && i && -1 && -i 
    \end{pmatrix}
    \begin{pmatrix}
        a_0 \\
        a_1 \\
        a_2 \\
        a_3
    \end{pmatrix}
\end{align}

または

\begin{align}
    \begin{cases}
        \tilde{a}_0 = (a_0 + a_2) + (a_1 + a_3) \\
        \tilde{a}_1 = (a_0 - a_2) - i(a_1 - a_3) \\
        \tilde{a}_2 = (a_0 + a_2) - (a_1 + a_3)\\   
        \tilde{a}_3 = (a_0 - a_2) + i(a_1 - a_3)
    \end{cases}
\end{align}
