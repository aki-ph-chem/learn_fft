\section{FFT}

ここでは、$N$ が2のべき乗である場合のみ考える。
離散フーリエ変換の式を偶数の項、奇数の項で分けて計算すると

\begin{align*}
    \tilde{a}_k &= \sum_{m = 0}^{N - 1} e^{-i\frac{2\pi}{N} km} a_m \\
                &= \sum_{m = 0}^{\frac{N}{2} - 1} e^{-i\frac{2\pi}{N} k(2m)} a_{2m} 
                   + \sum_{m = 0}^{\frac{N}{2} - 1} e^{-i\frac{2\pi}{N} k(2m + 1)} a_{2m + 1} \\ 
                &= \sum_{m = 0}^{\frac{N}{2} - 1} e^{-i\frac{2\pi}{N} k(2m)} a_{2m} 
                   + e^{-i\frac{2\pi}{N}k} \sum_{m = 0}^{\frac{N}{2} - 1} e^{-i\frac{2\pi}{N} k(2m)} a_{2m + 1} \\ 
                &= \sum_{m = 0}^{\frac{N}{2} - 1} e^{-i\frac{2\pi}{N/2} km} a_{2m} 
                   + e^{-i\frac{2\pi}{N}k} \sum_{m = 0}^{\frac{N}{2} - 1} e^{-i\frac{2\pi}{N/2} km} a_{2m + 1} \\ 
                &= \tilde{a}_k^{e} + W_N^{k} \tilde{a}_k^{o}
\end{align*}

より

\begin{align}\label{merge_1}
    \tilde{a}_k &= \tilde{a}_k^{e} + W_N^{k} \tilde{a}_k^{o}
\end{align}

が得られる。

ここで $W_N$ は

\begin{align*}
    W_N \equiv e^{-i\frac{2\pi}{N}} 
\end{align*}

で定義され、上付き添字の$e$,$o$はそれぞれ偶数・奇数の項であることを表す。

この偶数・奇数の項はもと長さ $N$ から $N/2$ となっている。
これを再帰的に繰り返し、長さが $1$ となったときには単なる恒等演算として返すようにすると、
偶数・奇数の項で離散フーリエ変換された列を得ることがでるので、最後に併合して一つの列とすれば離散フーリエ変換を構成することができる。

一つの列を偶数、奇数でそれぞれ長さ $N/2$ で分割したが、これを併合するには $0 \le k <  N/2 - 1$のときは(\ref{merge_1})を使えば良いが、 $N/2 \le < k < N - 1$ のときのは $\tilde{a}_{k + N/2}$ を計算する必要がある。 
これを計算すると


% 偶数・奇数をマージするときに必要な式

\begin{align*}
    \tilde{a}_{k + N/2} &= \sum_{m = 0}^{N - 1} e^{-i\frac{2\pi}{N} (k + N/2)m} a_m \\
                        &= \sum_{m = 0}^{N - 1} e^{-i\frac{2\pi}{N}km} e^{-i\frac{2\pi}{N}(N/2)m}  a_m \\
                        &= \sum_{m = 0}^{N - 1} e^{-i\frac{2\pi}{N}km} e^{-i\pi m}  a_m \\
\end{align*}

ここで、$e^{-i\pi m}$ は

\begin{align*}
    e^{-i\pi m} &= \cos(m\pi) - i\sin(m\pi)\\
                &= (-1)^m
\end{align*}

であるので、

\begin{align*}
    \tilde{a}_{k + N/2} &= \sum_{m = 0}^{N - 1} e^{-i\frac{2\pi}{N}km} (-1)^m a_m \\
                        &= \sum_{m = 0}^{\frac{N}{2} - 1} e^{-i\frac{2\pi}{N}k(2m)} a_{2m} 
                         - \sum_{m = 0}^{\frac{N}{2} - 1} e^{-i\frac{2\pi}{N}k(2m + 1)} a_{2m + 1} \\
                        &= \sum_{m = 0}^{\frac{N}{2} - 1} e^{-i\frac{2\pi}{N / 2}km} a_{2m} 
                         - \sum_{m = 0}^{\frac{N}{2} - 1} e^{-i\frac{2\pi}{N}k(2m + 1)} a_{2m + 1} \\
                        &= \sum_{m = 0}^{\frac{N}{2} - 1} e^{-i\frac{2\pi}{N / 2}km} a_{2m} 
                         - e^{-i\frac{2\pi}{N}k} \sum_{m = 0}^{\frac{N}{2} - 1} e^{-i\frac{2\pi}{N / 2}km} a_{2m + 1} \\
                        &= \sum_{m = 0}^{\frac{N}{2} - 1} e^{-i\frac{2\pi}{N / 2}km} a_{2m} 
                         - W_N^k \sum_{m = 0}^{\frac{N}{2} - 1} e^{-i\frac{2\pi}{N / 2}km} a_{2m + 1} \\
                        &= \tilde{a}_k^{e} - W_N^k \tilde{a}_k^{o}
\end{align*}

\begin{align}\label{merge_2}
    \tilde{a}_{k + N/2} &= \tilde{a}_k^{e} - W_N^k \tilde{a}_k^{o}
\end{align}

と得られるので(\ref{merge_2})によって行う。
