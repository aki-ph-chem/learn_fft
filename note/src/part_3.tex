\section{FFT}

ここでは、$N$ が2のべき乗である場合のみ考える。
離散フーリエ変換の式を偶数の項、奇数の項で分けて計算すると

\begin{align*}
    \tilde{a}_k &= \sum_{m = 0}^{N - 1} e^{-i\frac{2\pi}{N} km} a_m \\
                &= \sum_{m = 0}^{\frac{N}{2} - 1} e^{-i\frac{2\pi}{N} k(2m)} a_{2m} 
                   + \sum_{m = 0}^{\frac{N}{2} - 1} e^{-i\frac{2\pi}{N} k(2m + 1)} a_{2m + 1} \\ 
                &= \sum_{m = 0}^{\frac{N}{2} - 1} e^{-i\frac{2\pi}{N} k(2m)} a_{2m} 
                   + e^{-i\frac{2\pi}{N}k} \sum_{m = 0}^{\frac{N}{2} - 1} e^{-i\frac{2\pi}{N} k(2m)} a_{2m + 1} \\ 
                &= \sum_{m = 0}^{\frac{N}{2} - 1} e^{-i\frac{2\pi}{N/2} km} a_{2m} 
                   + e^{-i\frac{2\pi}{N}k} \sum_{m = 0}^{\frac{N}{2} - 1} e^{-i\frac{2\pi}{N/2} km} a_{2m + 1} \\ 
                &= \sum_{m = 0}^{\frac{N}{2} - 1} W_{N/2}^{km} a^{e}_{m} 
            + W_N^{k} \sum_{m = 0}^{\frac{N}{2} - 1} W_{N/2}^{km} a^{o}_{m} \\ 
\end{align*}

より

\begin{align}\label{merge_1}
    \tilde{a}_k &= \sum_{m = 0}^{\frac{N}{2} - 1} W_{N/2}^{km} a^{e}_{m} 
            + W_N^{k} \sum_{m = 0}^{\frac{N}{2} - 1} W_{N/2}^{km} a^{o}_{m} 
\end{align}

が得られる。

ここで $W_N$ は

\begin{align*}
    W_N \equiv e^{-i\frac{2\pi}{N}} 
\end{align*}

で定義され、右辺の $a_m$ 上付き添字の $e,o$ は偶数番目、奇数番目を取り出した部分列であることを意味する。

この左辺の第一項,第二項はそれぞれはもと長さ $N$ から $N/2$ についての和となっている。
これを再帰的に繰り返し、長さが $1$ となったときには単なる恒等演算として返すようにして、偶数・奇数の項を併合して一つの列とすれば離散フーリエ変換を構成することができる。

分割された列の併合は(\ref{merge_1})のみで $k$ を $0 \le k < N$ まで動かすループでも可能である。
しかし、高速化の方法の一つとしてループを展開する方法がある。ここでは $0 \le k < N$のループを $0 \le k < N/2 - 1$ のループで併合処理を $\tilde{a}_k$ と $\tilde{a}_{k + N/2}$ に分割することを考える。$\tilde{a}_{k + N/2}$ の計算を行うと 


% ループを2つに展開するのは結構キレイに書けるけど4つ、8つはあまりキレイではない(書けないことはないが...)

\begin{align*}
    \tilde{a}_{k + N/2} &= \sum_{m = 0}^{N - 1} e^{-i\frac{2\pi}{N} (k + N/2)m} a_m \\
                        &= \sum_{m = 0}^{N - 1} e^{-i\frac{2\pi}{N}km} e^{-i\frac{2\pi}{N}(N/2)m}  a_m \\
                        &= \sum_{m = 0}^{N - 1} e^{-i\frac{2\pi}{N}km} e^{-i\pi m}  a_m \\
\end{align*}

ここで、$e^{-i\pi m}$ は

\begin{align*}
    e^{-i\pi m} &= \cos(m\pi) - i\sin(m\pi)\\
                &= (-1)^m
\end{align*}

であるので、

\begin{align*}
    \tilde{a}_{k + N/2} &= \sum_{m = 0}^{N - 1} e^{-i\frac{2\pi}{N}km} (-1)^m a_m \\
                        &= \sum_{m = 0}^{\frac{N}{2} - 1} e^{-i\frac{2\pi}{N}k(2m)} a_{2m} 
                         - \sum_{m = 0}^{\frac{N}{2} - 1} e^{-i\frac{2\pi}{N}k(2m + 1)} a_{2m + 1} \\
                        &= \sum_{m = 0}^{\frac{N}{2} - 1} e^{-i\frac{2\pi}{N / 2}km} a_{2m} 
                         - \sum_{m = 0}^{\frac{N}{2} - 1} e^{-i\frac{2\pi}{N}k(2m + 1)} a_{2m + 1} \\
                        &= \sum_{m = 0}^{\frac{N}{2} - 1} e^{-i\frac{2\pi}{N / 2}km} a_{2m} 
                         - e^{-i\frac{2\pi}{N}k} \sum_{m = 0}^{\frac{N}{2} - 1} e^{-i\frac{2\pi}{N / 2}km} a_{2m + 1} \\
                        &= \sum_{m = 0}^{\frac{N}{2} - 1} W_{N/2}^{km} a^{e}_{m} 
                        - W_N^k \sum_{m = 0}^{\frac{N}{2} - 1} W_{N/2}^{km} a^{o}_{m} \\
\end{align*}

\begin{align}\label{merge_2}
    \tilde{a}_{k + N/2} &= \sum_{m = 0}^{\frac{N}{2} - 1} W_{N/2}^{km} a^{e}_{m} 
                        - W_N^k \sum_{m = 0}^{\frac{N}{2} - 1} W_{N/2}^{km} a^{o}_{m}
\end{align}

と得られた(\ref{merge_2})によって行う。
